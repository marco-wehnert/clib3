\documentclass{stddoc}

\title{CLIB3 Developers Guide}
\author{Marco Wehnert}

\begin{document}

\maketitle

\section{Coding Style}
	If nothing else is defined the coding style will be the same like
	in the Linux kernel. (see \url{https://www.kernel.org/doc/html/latest/process/coding-style.html})


\section{Naming Convention}

	\begin{itemize}
		\item All items shall start with CL3\_
		\item Don't create typedefs for structues (There are a few exceptions
			which are mentioned in the coding style for the Linux kernel).
	\end{itemize}


\section{Object-oriented programming in C}
	Sometimes it is helpful to design a software similar to object-oriented
	style. The problem is that this is not possible only using C. But ...
	we can implement a "pseudo object-oriented" model. To do so we
	just put all member variables of the class in a structure and pass
	the structure as first argument in a function. And voila, we have
	an object-oriented implementation only using C.
	In such cases the function names should
	start with the name of the structure followed by the function name
	to avoid naming conflicts for functions.
	Functions which are not relatedd to class will be not affeted by this
	rule.


\section{Configuration}
	There has to be the possibility to configure the components of this
	library (using for example a \#define). For this each header file will
	include the file "cl3config.h"

\section{Components}
\subsection{TCP Message Protocol}
	Per definition TCP is a stream based protocol, there is no mechanism
	to determine where one message ends and where the next message begins.
	This mechanism needs to be implemented in the application itself.
	The implementation here provides such a mechanism.

	A message is defined according to table x.

	\begin{longtable}{|L{1.5cm}|L{0.8cm}|L{0.8cm}|L{9cm}|L{1.8cm}|}
		\caption{Header format of TCP message} \\
		\hline
		\textbf{Item} & Pos. & Len. & Description & Values \\
		\hline \hline
		Sync pattern & 0 & 2 & Sync pattern & 0xAA \\
		\hline
	\end{longtable}

	\begin{itemize}
		\item Prefix for all items is \code{CL3\_TcpConnection\_}
	\end{itemize}
\end{document}
